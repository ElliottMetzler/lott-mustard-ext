\documentclass{article}

\usepackage{amsmath}
\usepackage{amssymb}

\usepackage{booktabs}
\usepackage{float}
\usepackage{colortbl}
\usepackage{xcolor}

\usepackage{a4wide}
\usepackage{setspace}
\usepackage{geometry}
\usepackage{pdflscape}
\usepackage{parskip}
\doublespacing
\geometry{margin=1.5in}

\usepackage{graphicx}
\graphicspath{ {../figures/} }

\usepackage{hyperref}
\hypersetup{
	colorlinks = true,
	linkcolor = black,
	urlcolor=blue
}

\author{Elliott Metzler}
\title{Lott and Mustard Replication Assignment}
\date{5/4/2022}

\begin{document}
\maketitle

\section{Introduction}

\textbf{Prompt:}  What is the purpose of your study?  State very clearly why people should care about this project.  

\section{Background and Economic Theory}

\textbf{Prompt:} This section should only focus on the original Lott and Mustard (1997) project.  What was this study?  How did they do it?  How were the laws coded?  It should include a table of the rollout state by state (Table 1), clear description of the laws, the theory behind why you should find deterrence (as well as a description of the economic theory of deterrence), and what the authors found in their original paper.  

\begin{table}[H]

\caption{\label{tab:tab:replicatetable1}Replication of Table 1}
\centering
\begin{tabular}[t]{lrr}
\toprule
Variable & Standard Deviation of State Means & Mean of Within-State Standard Deviations\\
\midrule
\cellcolor{gray!6}{Violent Crime Rate} & \cellcolor{gray!6}{306.62} & \cellcolor{gray!6}{78.19}\\
Property Crime Rate & 1135.99 & 422.15\\
\cellcolor{gray!6}{Murder Crime Rate} & \cellcolor{gray!6}{6.19} & \cellcolor{gray!6}{1.50}\\
Rape Crime Rate & 13.26 & 6.37\\
\cellcolor{gray!6}{Assault Crime Rate} & \cellcolor{gray!6}{144.28} & \cellcolor{gray!6}{57.62}\\
\addlinespace
Robbery Crime Rate & 171.87 & 28.81\\
\cellcolor{gray!6}{Burglary Crime Rate} & \cellcolor{gray!6}{372.52} & \cellcolor{gray!6}{181.39}\\
Larceny Crime Rate & 705.73 & 260.55\\
\cellcolor{gray!6}{Autotheft Crime Rate} & \cellcolor{gray!6}{210.57} & \cellcolor{gray!6}{82.24}\\
Violent Arrest Rate & 10.38 & 10.54\\
\addlinespace
\cellcolor{gray!6}{Property Arrest Rate} & \cellcolor{gray!6}{3.07} & \cellcolor{gray!6}{3.23}\\
Murder Arrest Rate & 17.26 & 29.23\\
\cellcolor{gray!6}{Rape Arrest Rate} & \cellcolor{gray!6}{11.15} & \cellcolor{gray!6}{10.65}\\
Assault Arrest Rate & 11.42 & 10.81\\
\cellcolor{gray!6}{Robery Arrest Rate} & \cellcolor{gray!6}{8.83} & \cellcolor{gray!6}{8.09}\\
\addlinespace
Burglary Arrest Rate & 3.08 & 3.20\\
\cellcolor{gray!6}{Larceny Arrest Rate} & \cellcolor{gray!6}{3.42} & \cellcolor{gray!6}{3.49}\\
Autotheft Arrest Rate & 10.67 & 31.62\\
\bottomrule
\end{tabular}
\end{table}


\section{Data}

\textbf{Prompt:} In this section, you will describe the state-level data.  While the authors used county-level data, we will be using state-level since the laws are ultimately at the state level and critics have noted that the county-level data has substantial measurement error.  Describe the data, produce a table of summary statistics for the various crime outcomes as Table 2. Anything additional that you want to show (like time series of the crimes) is up to you. The goal is to create a readable document, so make your own choices.

\begin{table}[H]

\caption{\label{tab:tab:replicatetable2a}Main Variables Summary}
\centering
\begin{tabular}[t]{lrrr}
\toprule
Variable & N & Mean & Standard Deviation\\
\midrule
\cellcolor{gray!6}{Shalll} & \cellcolor{gray!6}{816} & \cellcolor{gray!6}{0.191} & \cellcolor{gray!6}{0.393}\\
Violent Arrest Rate & 802 & 41.091 & 22.204\\
\cellcolor{gray!6}{Property Arrest Rate} & \cellcolor{gray!6}{809} & \cellcolor{gray!6}{16.918} & \cellcolor{gray!6}{4.677}\\
Murder Arrest Rate & 806 & 91.299 & 55.943\\
\cellcolor{gray!6}{Rape Arrest Rate} & \cellcolor{gray!6}{799} & \cellcolor{gray!6}{41.023} & \cellcolor{gray!6}{17.389}\\
\addlinespace
Assault Arrest Rate & 809 & 44.625 & 16.978\\
\cellcolor{gray!6}{Robery Arrest Rate} & \cellcolor{gray!6}{808} & \cellcolor{gray!6}{31.458} & \cellcolor{gray!6}{13.593}\\
Burglary Arrest Rate & 809 & 13.804 & 4.571\\
\cellcolor{gray!6}{Larceny Arrest Rate} & \cellcolor{gray!6}{809} & \cellcolor{gray!6}{18.537} & \cellcolor{gray!6}{5.196}\\
Autotheft Arrest Rate & 808 & 22.345 & 37.611\\
\addlinespace
\cellcolor{gray!6}{Violent Crime Rate} & \cellcolor{gray!6}{816} & \cellcolor{gray!6}{483.926} & \cellcolor{gray!6}{318.943}\\
Property Crime Rate & 816 & 4618.339 & 1210.465\\
\cellcolor{gray!6}{Murder Crime Rate} & \cellcolor{gray!6}{816} & \cellcolor{gray!6}{7.768} & \cellcolor{gray!6}{6.882}\\
Rape Crime Rate & 816 & 33.982 & 15.072\\
\cellcolor{gray!6}{Assault Crime Rate} & \cellcolor{gray!6}{816} & \cellcolor{gray!6}{278.755} & \cellcolor{gray!6}{159.650}\\
\addlinespace
Robbery Crime Rate & 816 & 163.421 & 176.251\\
\cellcolor{gray!6}{Burglary Crime Rate} & \cellcolor{gray!6}{816} & \cellcolor{gray!6}{1239.336} & \cellcolor{gray!6}{417.758}\\
Larceny Crime Rate & 816 & 2968.708 & 751.023\\
\cellcolor{gray!6}{Autotheft Crime Rate} & \cellcolor{gray!6}{816} & \cellcolor{gray!6}{410.295} & \cellcolor{gray!6}{231.154}\\
Personal Income Rpc & 816 & 9351.821 & 4689.701\\
\addlinespace
\cellcolor{gray!6}{Unemployment Insurance Rpc} & \cellcolor{gray!6}{816} & \cellcolor{gray!6}{50.019} & \cellcolor{gray!6}{38.081}\\
Income Maintenance Rpc & 816 & 115.276 & 70.953\\
\cellcolor{gray!6}{Retirement Payments Rpc} & \cellcolor{gray!6}{816} & \cellcolor{gray!6}{1002.226} & \cellcolor{gray!6}{546.468}\\
State Population & 816 & 4646787.342 & 5010349.873\\
\cellcolor{gray!6}{Density} & \cellcolor{gray!6}{816} & \cellcolor{gray!6}{355.973} & \cellcolor{gray!6}{1408.250}\\
\bottomrule
\end{tabular}
\end{table}


\begin{table}[H]

\caption{\label{tab:tab:replicatetable2b}Demographic Variables Summary}
\centering
\begin{tabular}[t]{lrrr}
\toprule
Variable & N & Mean & Standard Deviation\\
\midrule
\cellcolor{gray!6}{White Male 1019} & \cellcolor{gray!6}{816} & \cellcolor{gray!6}{0.067} & \cellcolor{gray!6}{0.015}\\
Black Male 1019 & 816 & 0.010 & 0.011\\
\cellcolor{gray!6}{Other Male 1019} & \cellcolor{gray!6}{816} & \cellcolor{gray!6}{0.004} & \cellcolor{gray!6}{0.008}\\
White Female 1019 & 816 & 0.064 & 0.015\\
\cellcolor{gray!6}{Black Female 1019} & \cellcolor{gray!6}{816} & \cellcolor{gray!6}{0.010} & \cellcolor{gray!6}{0.011}\\
\addlinespace
Other Female 1019 & 816 & 0.004 & 0.007\\
\cellcolor{gray!6}{White Male 2029} & \cellcolor{gray!6}{816} & \cellcolor{gray!6}{0.074} & \cellcolor{gray!6}{0.012}\\
Black Male 2029 & 816 & 0.010 & 0.010\\
\cellcolor{gray!6}{Other Male 2029} & \cellcolor{gray!6}{816} & \cellcolor{gray!6}{0.004} & \cellcolor{gray!6}{0.007}\\
White Female 2029 & 816 & 0.073 & 0.012\\
\addlinespace
\cellcolor{gray!6}{Black Female 2029} & \cellcolor{gray!6}{816} & \cellcolor{gray!6}{0.010} & \cellcolor{gray!6}{0.012}\\
Other Female 2029 & 816 & 0.004 & 0.007\\
\cellcolor{gray!6}{White Male 3039} & \cellcolor{gray!6}{816} & \cellcolor{gray!6}{0.066} & \cellcolor{gray!6}{0.012}\\
Black Male 3039 & 816 & 0.007 & 0.008\\
\cellcolor{gray!6}{Other Male 3039} & \cellcolor{gray!6}{816} & \cellcolor{gray!6}{0.003} & \cellcolor{gray!6}{0.007}\\
\addlinespace
White Female 3039 & 816 & 0.066 & 0.012\\
\cellcolor{gray!6}{Black Female 3039} & \cellcolor{gray!6}{816} & \cellcolor{gray!6}{0.008} & \cellcolor{gray!6}{0.010}\\
Other Female 3039 & 816 & 0.003 & 0.007\\
\cellcolor{gray!6}{White Male 4049} & \cellcolor{gray!6}{816} & \cellcolor{gray!6}{0.048} & \cellcolor{gray!6}{0.009}\\
Black Male 4049 & 816 & 0.005 & 0.006\\
\addlinespace
\cellcolor{gray!6}{Other Male 4049} & \cellcolor{gray!6}{816} & \cellcolor{gray!6}{0.002} & \cellcolor{gray!6}{0.005}\\
White Female 4049 & 816 & 0.048 & 0.009\\
\cellcolor{gray!6}{Black Female 4049} & \cellcolor{gray!6}{816} & \cellcolor{gray!6}{0.005} & \cellcolor{gray!6}{0.007}\\
Other Female 4049 & 816 & 0.002 & 0.005\\
\cellcolor{gray!6}{White Male 5064} & \cellcolor{gray!6}{816} & \cellcolor{gray!6}{0.058} & \cellcolor{gray!6}{0.010}\\
\addlinespace
Black Male 5064 & 816 & 0.005 & 0.007\\
\cellcolor{gray!6}{Other Male 5064} & \cellcolor{gray!6}{816} & \cellcolor{gray!6}{0.002} & \cellcolor{gray!6}{0.006}\\
White Female 5064 & 816 & 0.062 & 0.012\\
\cellcolor{gray!6}{Black Female 5064} & \cellcolor{gray!6}{816} & \cellcolor{gray!6}{0.006} & \cellcolor{gray!6}{0.009}\\
Other Female 5064 & 816 & 0.002 & 0.007\\
\addlinespace
\cellcolor{gray!6}{White Male 65o} & \cellcolor{gray!6}{816} & \cellcolor{gray!6}{0.043} & \cellcolor{gray!6}{0.011}\\
Black Male 65o & 816 & 0.004 & 0.005\\
\cellcolor{gray!6}{Other Male 65o} & \cellcolor{gray!6}{816} & \cellcolor{gray!6}{0.001} & \cellcolor{gray!6}{0.005}\\
White Female 65o & 816 & 0.062 & 0.017\\
\cellcolor{gray!6}{Black Female 65o} & \cellcolor{gray!6}{816} & \cellcolor{gray!6}{0.005} & \cellcolor{gray!6}{0.008}\\
\addlinespace
Other Female 65o & 816 & 0.001 & 0.005\\
\bottomrule
\end{tabular}
\end{table}



\section{Empirical Model and Estimation}

\begin{itemize}
\item 
Explain what this section is going to be about at a high level
\item
Quickly Summarize anything interesting

\end{itemize}

\subsection{Two way Fixed Effects}

\begin{itemize}
\item 
Describe the model used by the authors
\item
Describe how this model is different
\item
Explain exclusion of controls from the tables but that they are analyzed
\item 
Draft explanation of model results
\end{itemize}


\begin{table}[htbp]
   \caption{\label{tab:replicatetable3a} Replication of Table 3 Panel A : Fixed Effects Regressions}
   \centering
   \small
   \begin{tabular}{lccc}
      \tabularnewline \midrule \midrule
      Dependent Variables:     & violent\_crime\_rate\_log    & property\_crime\_rate\_log    & murder\_crime\_rate\_log\\     
      Model:                   & (1)                          & (2)                           & (3)\\  
      \midrule
      \emph{Variables}\\
      shalll                   & -0.0978$^{***}$              & -0.0072                       & -0.0507\\   
                               & (0.0324)                     & (0.0194)                      & (0.0394)\\   
      Relevant\_Arrest\_Rate   & -0.0003                      & -0.0020$^{*}$                 & -0.0004\\   
                               & (0.0004)                     & (0.0011)                      & (0.0002)\\   
      \midrule
      \emph{Fixed-effects}\\
      state                    & Yes                          & Yes                           & Yes\\  
      year                     & Yes                          & Yes                           & Yes\\  
      \midrule
      \emph{Fit statistics}\\
      Observations             & 802                          & 809                           & 806\\  
      R$^2$                    & 0.98146                      & 0.96445                       & 0.94792\\  
      Within R$^2$             & 0.47334                      & 0.55426                       & 0.31137\\  
      \midrule \midrule
      \multicolumn{4}{l}{\emph{Clustered (state) standard-errors in parentheses}}\\
      \multicolumn{4}{l}{\emph{Signif. Codes: ***: 0.01, **: 0.05, *: 0.1}}\\
   \end{tabular}
   
   \par \raggedright 
   Control variables ommited from table, 
                           though they were included in the analysis. 
                           Consistent with the original paper, 
                           control variables include: density, personal\_income\_rpc, unemployment\_insurance\_rpc, income\_maintenance\_rpc, retirement\_payments\_rpc, state\_population, ppwm1019, ppbm1019, ppnm1019, ppwf1019, ppbf1019, ppnf1019, ppwm2029, ppbm2029, ppnm2029, ppwf2029, ppbf2029, ppnf2029, ppwm3039, ppbm3039, ppnm3039, ppwf3039, ppbf3039, ppnf3039, ppwm4049, ppbm4049, ppnm4049, ppwf4049, ppbf4049, ppnf4049, ppwm5064, ppbm5064, ppnm5064, ppwf5064, ppbf5064, ppnf5064, ppwm65o, ppbm65o, ppnm65o, ppwf65o, ppbf65o, ppnf65o.
\end{table}





\begin{table}[htbp]
   \caption{\label{tab:replicatetable3b} Replication of Table 3 Panel B : Fixed Effects Regressions}
   \centering
   \small
   \begin{tabular}{lccc}
      \tabularnewline \midrule \midrule
      Dependent Variables:     & rape\_crime\_rate\_log    & assault\_crime\_rate\_log    & robbery\_crime\_rate\_log\\     
      Model:                   & (1)                       & (2)                          & (3)\\  
      \midrule
      \emph{Variables}\\
      shalll                   & -0.0340                   & -0.1004$^{**}$               & -0.0532\\   
                               & (0.0395)                  & (0.0424)                     & (0.0439)\\   
      Relevant\_Arrest\_Rate   & -0.0006                   & -0.0028$^{***}$              & -0.0014\\   
                               & (0.0005)                  & (0.0009)                     & (0.0009)\\   
      \midrule
      \emph{Fixed-effects}\\
      state                    & Yes                       & Yes                          & Yes\\  
      year                     & Yes                       & Yes                          & Yes\\  
      \midrule
      \emph{Fit statistics}\\
      Observations             & 799                       & 809                          & 808\\  
      R$^2$                    & 0.94240                   & 0.96632                      & 0.98389\\  
      Within R$^2$             & 0.52351                   & 0.51645                      & 0.50471\\  
      \midrule \midrule
      \multicolumn{4}{l}{\emph{Clustered (state) standard-errors in parentheses}}\\
      \multicolumn{4}{l}{\emph{Signif. Codes: ***: 0.01, **: 0.05, *: 0.1}}\\
   \end{tabular}
   
   \par \raggedright 
   Control variables ommited from table, 
                           though they were included in the analysis. 
                           Consistent with the original paper, 
                           control variables include: density, personal\_income\_rpc, unemployment\_insurance\_rpc, income\_maintenance\_rpc, retirement\_payments\_rpc, state\_population, ppwm1019, ppbm1019, ppnm1019, ppwf1019, ppbf1019, ppnf1019, ppwm2029, ppbm2029, ppnm2029, ppwf2029, ppbf2029, ppnf2029, ppwm3039, ppbm3039, ppnm3039, ppwf3039, ppbf3039, ppnf3039, ppwm4049, ppbm4049, ppnm4049, ppwf4049, ppbf4049, ppnf4049, ppwm5064, ppbm5064, ppnm5064, ppwf5064, ppbf5064, ppnf5064, ppwm65o, ppbm65o, ppnm65o, ppwf65o, ppbf65o, ppnf65o.
\end{table}





\begin{table}[htbp]
   \caption{\label{tab:replicatetable3c} Replication of Table 3 Panel C : Fixed Effects Regressions}
   \centering
   \small
   \begin{tabular}{lccc}
      \tabularnewline \midrule \midrule
      Dependent Variables:     & burglary\_crime\_rate\_log    & larceny\_crime\_rate\_log    & autotheft\_crime\_rate\_log\\     
      Model:                   & (1)                           & (2)                          & (3)\\  
      \midrule
      \emph{Variables}\\
      shalll                   & -0.0461$^{**}$                & 0.0033                       & -0.0090\\   
                               & (0.0190)                      & (0.0138)                     & (0.0283)\\   
      Relevant\_Arrest\_Rate   & -0.0052$^{***}$               & -0.0011                      & -0.0003$^{*}$\\   
                               & (0.0015)                      & (0.0010)                     & (0.0002)\\   
      \midrule
      \emph{Fixed-effects}\\
      state                    & Yes                           & Yes                          & Yes\\  
      year                     & Yes                           & Yes                          & Yes\\  
      \midrule
      \emph{Fit statistics}\\
      Observations             & 809                           & 809                          & 808\\  
      R$^2$                    & 0.95597                       & 0.96600                      & 0.96116\\  
      Within R$^2$             & 0.50429                       & 0.54529                      & 0.60312\\  
      \midrule \midrule
      \multicolumn{4}{l}{\emph{Heteroskedasticity-robust standard-errors in parentheses}}\\
      \multicolumn{4}{l}{\emph{Signif. Codes: ***: 0.01, **: 0.05, *: 0.1}}\\
   \end{tabular}
   
   \par \raggedright 
   Control variables ommited from table, 
                           though they were included in the analysis. 
                           Consistent with the original paper, 
                           control variables include: density, personal\_income\_rpc, unemployment\_insurance\_rpc, income\_maintenance\_rpc, retirement\_payments\_rpc, state\_population, ppwm1019, ppbm1019, ppnm1019, ppwf1019, ppbf1019, ppnf1019, ppwm2029, ppbm2029, ppnm2029, ppwf2029, ppbf2029, ppnf2029, ppwm3039, ppbm3039, ppnm3039, ppwf3039, ppbf3039, ppnf3039, ppwm4049, ppbm4049, ppnm4049, ppwf4049, ppbf4049, ppnf4049, ppwm5064, ppbm5064, ppnm5064, ppwf5064, ppbf5064, ppnf5064, ppwm65o, ppbm65o, ppnm65o, ppwf65o, ppbf65o, ppnf65o.
\end{table}




\subsection{Bacon Decomposition}

\begin{itemize}
\item
Explain Analysis: Bacon Decomposition without controls
\item
Explain weights and DiD estimates
\item
Explain early to late 2x2s and late to early 2x2s. What is the problem with late to early?
\end{itemize}

\begin{landscape}
\begin{table}[H]

\caption{\label{tab:tab:bacondecompositiona}Bacon Decomposition: Panel A}
\centering
\begin{tabular}[t]{lrrr}
\toprule
Type & Violent crime rate log & Property crime rate log & Murder crime rate log\\
\midrule
\cellcolor{gray!6}{Earlier vs Later Treated} & \cellcolor{gray!6}{0.0051705} & \cellcolor{gray!6}{-0.0007191} & \cellcolor{gray!6}{0.0054525}\\
Later vs Always Treated & -0.0044690 & 0.0085564 & -0.0012542\\
\cellcolor{gray!6}{Later vs Earlier Treated} & \cellcolor{gray!6}{-0.0017883} & \cellcolor{gray!6}{0.0001507} & \cellcolor{gray!6}{0.0000418}\\
Treated vs Untreated & -0.0839315 & 0.0210569 & -0.0415959\\
\cellcolor{gray!6}{Total TWFE} & \cellcolor{gray!6}{-0.0850183} & \cellcolor{gray!6}{0.0290449} & \cellcolor{gray!6}{-0.0373558}\\
\bottomrule
\end{tabular}
\end{table}


\begin{table}[H]

\caption{\label{tab:tab:bacondecompositionb}Bacon Decomposition: Panel B}
\centering
\begin{tabular}[t]{lrrr}
\toprule
Type & Rape crime rate log & Assault crime rate log & Robbery crime rate log\\
\midrule
\cellcolor{gray!6}{Earlier vs Later Treated} & \cellcolor{gray!6}{-0.0026425} & \cellcolor{gray!6}{0.0079628} & \cellcolor{gray!6}{0.0073682}\\
Later vs Always Treated & -0.0305273 & 0.0009728 & 0.0174822\\
\cellcolor{gray!6}{Later vs Earlier Treated} & \cellcolor{gray!6}{-0.0019283} & \cellcolor{gray!6}{-0.0034427} & \cellcolor{gray!6}{0.0020947}\\
Treated vs Untreated & 0.0030423 & -0.1375213 & -0.0100514\\
\cellcolor{gray!6}{Total TWFE} & \cellcolor{gray!6}{-0.0320558} & \cellcolor{gray!6}{-0.1320284} & \cellcolor{gray!6}{0.0168936}\\
\bottomrule
\end{tabular}
\end{table}


\begin{table}[H]

\caption{\label{tab:tab:bacondecompositionc}Bacon Decomposition: Panel C}
\centering
\begin{tabular}[t]{lrrr}
\toprule
Type & Burglary crime rate log & Larceny crime rate log & Autotheft crime rate log\\
\midrule
\cellcolor{gray!6}{Earlier vs Later Treated} & \cellcolor{gray!6}{-0.0023226} & \cellcolor{gray!6}{-0.0004160} & \cellcolor{gray!6}{0.0056843}\\
Later vs Always Treated & 0.0048413 & 0.0080003 & 0.0336117\\
\cellcolor{gray!6}{Later vs Earlier Treated} & \cellcolor{gray!6}{-0.0013013} & \cellcolor{gray!6}{0.0004859} & \cellcolor{gray!6}{0.0020305}\\
Treated vs Untreated & 0.0064291 & 0.0280699 & 0.0262756\\
\cellcolor{gray!6}{Total TWFE} & \cellcolor{gray!6}{0.0076465} & \cellcolor{gray!6}{0.0361400} & \cellcolor{gray!6}{0.0676020}\\
\bottomrule
\end{tabular}
\end{table}


\end{landscape}


\subsection{Callaway and Sant'anna}

\textbf{Prompt: } Present a subsection in which you implement the Callaway and Sant’anna estimator.  Describe the model with an equation and a description (short). Use the double robust specification. You will be analyzing each outcome and reporting the overall ATT.  Do not report the group-level ATTs because many states simply do not have enough states per treatment date for the bootstrapping to provide accurate 95 percent confidence intervals.  Use no more than 2 covariates – use your own judgment in selectin them. Report this in Table 4 and in your discussion compare what you found with the original findings.  Are they similar?  If not how do they differ?

\subsection{Event Study (Sun and Abraham)}

\textbf{Prompt: } Finally, implement the Sun and Abraham event study.  While you can estimate Callaway and Sant’anna event studies, I would like to use Sun and Abraham.  Explain the interaction weighted estimator and show a figure of each crime.  Do pretrends appear to hold?  How confident do you feel then that parallel trends holds for each outcome.  This should only be presented as a Figure, not a table.  

\section{Conclusion}

\textbf{Prompt: }What do you think you learned from this exercise?  Feel free to discuss as little or as much as you want.  I am just interested in your opinions.  The purpose of this is merely to give you a nudge in considering how to interpret results and offer some commentary. 

\end{document}

