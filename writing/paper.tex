\documentclass{article}

\usepackage{amsmath}
\usepackage{amssymb}

\usepackage{booktabs}
\usepackage{float}
\usepackage{colortbl}
\usepackage{xcolor}

\usepackage{a4wide}
\usepackage{setspace}
\usepackage{geometry}
\usepackage{pdflscape}
\usepackage{parskip}
\doublespacing
\geometry{margin=1.5in}

\usepackage{graphicx}
\graphicspath{ {../figures/} }

\usepackage{hyperref}
\hypersetup{
	colorlinks = true,
	linkcolor = black,
	urlcolor=blue
}

\author{Elliott Metzler}
\title{Lott and Mustard Replication Assignment}
\date{5/4/2022}

\begin{document}
\maketitle

\section{Introduction}

This paper explores the findings of Lott and Mustard's 1997 paper titled ``Crime, Deterrence, and Right-to-Carry Concealed Handguns''. Specifically, we explore the original findings of the paper through a replication analysis using state level data, and extend the analysis to include more modern techniques of causal inference. Through application of modern techniques, we are able to better understand the original findings and ascertain their validity with more certainty. 

In the original paper, the writers focused on applying a Fixed Effects model to their data. We replicate this model at the state level for comparison's sake. Additionally, we apply the Bacon Decomposition in order to more rigorously examine the underlying weights and pieces of the fixed effects coefficients produced by the original model. Third, we implement the Callaway and Sant'anna estimator. Last, we implement the Sun and Abraham event study. 

The remainder of this paper is structured as follows. First, we discuss the background and economic theory of the original paper. Next, we discuss the data used for the replication and application of additional estimators. Third, we will present the four empirical models we implement on the data, discuss their purpose and implications, and present our results. Finally, we conclude with a summary and some implications of our study.

\section{Background and Economic Theory}

The original paper had a few key aims and implications. Principally, the authors were attempting to assess the impact of changes in concealed carry laws in the United States. Their study used data at the county level from the years 1977 to 1992 and included information on arrest rates, crime rates, demographic data, some economic data, and most importantly, an indicator variable for whether or not the county had a ``shall issue'' law. The ``shall issue'' law was useful for their purpose because these laws require officials to issue conceal carry gun permits to anyone who passes a basic screen for criminal record or history of significant mental illness. With this indicator variable, they posited that they could identify the causal impact of concealed carry on crime deterrence. We display the years in which each state that issued a shall issue law between 1977 and 1992 in Table 1.

\begin{table}[H]

\caption{\label{tab:tab:rollout}Shall Issue Law Rollout By State}
\centering
\begin{tabular}[t]{lr}
\toprule
State & Year\\
\midrule
\cellcolor{gray!6}{Alabama} & \cellcolor{gray!6}{1977}\\
Connecticut & 1977\\
\cellcolor{gray!6}{New Hampshire} & \cellcolor{gray!6}{1977}\\
North Dakota & 1977\\
\cellcolor{gray!6}{South Dakota} & \cellcolor{gray!6}{1977}\\
\addlinespace
Vermont & 1977\\
\cellcolor{gray!6}{Washington} & \cellcolor{gray!6}{1977}\\
Indiana & 1981\\
\cellcolor{gray!6}{Maine} & \cellcolor{gray!6}{1986}\\
Florida & 1988\\
\addlinespace
\cellcolor{gray!6}{Virginia} & \cellcolor{gray!6}{1989}\\
Georgia & 1990\\
\cellcolor{gray!6}{Pennsylvania} & \cellcolor{gray!6}{1990}\\
West Virginia & 1990\\
\cellcolor{gray!6}{Idaho} & \cellcolor{gray!6}{1991}\\
\addlinespace
Mississippi & 1991\\
\cellcolor{gray!6}{Oregon} & \cellcolor{gray!6}{1991}\\
Montana & 1992\\
\bottomrule
\end{tabular}
\end{table}


The author's main analytical approach was to use a two-way fixed effects model, accounting for as much variation between units (counties) as possible to isolate the impact of the shall issue laws on various crime rates. More specifically, the authors regressed the natural log of crime rate on a dummy for the shall issue law, the arrest rate for the same crime category in question, some economic-related variables (population per square mile, unemployment insurance, etc.), and demographic distribution variables. The crimes they evaluated included murder, rape, aggravated assault, robbery, property crime, burglary, larceny, and auto theft. Additionally, they combined murder, rape, aggravated assault, and robbery into a category ``violent crime'' and the other three into a category called ``property crimes.'' For each of these crime categories, they estimated the two-way fixed effects model.

The author's main results from this approach show that shall issue laws are negatively related to each of the violent crimes. They also find that the shall issue laws are negatively associated with the property crimes. 

\textbf{Need to do for this section:}
\begin{enumerate}
\item 
Add explanation of deterrence
\end{enumerate}

\section{Data}

The data used for this replication and extension analysis is at the state level and includes each of the original variables required to replicate the main two-way fixed effects analysis performed by Lott and Mustard. Importantly, the data includes a row for each state for each year and each of the arrest rate, crime rate, economic, demographic, and shall issue indicator variables. We present summary statistics of the variables in Table 2 and Table 3. 

\begin{table}[H]

\caption{\label{tab:tab:replicatetable2a}Main Variables Summary}
\centering
\begin{tabular}[t]{lrrr}
\toprule
Variable & N & Mean & Standard Deviation\\
\midrule
\cellcolor{gray!6}{Shalll} & \cellcolor{gray!6}{816} & \cellcolor{gray!6}{0.191} & \cellcolor{gray!6}{0.393}\\
Violent Arrest Rate & 802 & 41.091 & 22.204\\
\cellcolor{gray!6}{Property Arrest Rate} & \cellcolor{gray!6}{809} & \cellcolor{gray!6}{16.918} & \cellcolor{gray!6}{4.677}\\
Murder Arrest Rate & 806 & 91.299 & 55.943\\
\cellcolor{gray!6}{Rape Arrest Rate} & \cellcolor{gray!6}{799} & \cellcolor{gray!6}{41.023} & \cellcolor{gray!6}{17.389}\\
\addlinespace
Assault Arrest Rate & 809 & 44.625 & 16.978\\
\cellcolor{gray!6}{Robery Arrest Rate} & \cellcolor{gray!6}{808} & \cellcolor{gray!6}{31.458} & \cellcolor{gray!6}{13.593}\\
Burglary Arrest Rate & 809 & 13.804 & 4.571\\
\cellcolor{gray!6}{Larceny Arrest Rate} & \cellcolor{gray!6}{809} & \cellcolor{gray!6}{18.537} & \cellcolor{gray!6}{5.196}\\
Autotheft Arrest Rate & 808 & 22.345 & 37.611\\
\addlinespace
\cellcolor{gray!6}{Violent Crime Rate} & \cellcolor{gray!6}{816} & \cellcolor{gray!6}{483.926} & \cellcolor{gray!6}{318.943}\\
Property Crime Rate & 816 & 4618.339 & 1210.465\\
\cellcolor{gray!6}{Murder Crime Rate} & \cellcolor{gray!6}{816} & \cellcolor{gray!6}{7.768} & \cellcolor{gray!6}{6.882}\\
Rape Crime Rate & 816 & 33.982 & 15.072\\
\cellcolor{gray!6}{Assault Crime Rate} & \cellcolor{gray!6}{816} & \cellcolor{gray!6}{278.755} & \cellcolor{gray!6}{159.650}\\
\addlinespace
Robbery Crime Rate & 816 & 163.421 & 176.251\\
\cellcolor{gray!6}{Burglary Crime Rate} & \cellcolor{gray!6}{816} & \cellcolor{gray!6}{1239.336} & \cellcolor{gray!6}{417.758}\\
Larceny Crime Rate & 816 & 2968.708 & 751.023\\
\cellcolor{gray!6}{Autotheft Crime Rate} & \cellcolor{gray!6}{816} & \cellcolor{gray!6}{410.295} & \cellcolor{gray!6}{231.154}\\
Personal Income Rpc & 816 & 9351.821 & 4689.701\\
\addlinespace
\cellcolor{gray!6}{Unemployment Insurance Rpc} & \cellcolor{gray!6}{816} & \cellcolor{gray!6}{50.019} & \cellcolor{gray!6}{38.081}\\
Income Maintenance Rpc & 816 & 115.276 & 70.953\\
\cellcolor{gray!6}{Retirement Payments Rpc} & \cellcolor{gray!6}{816} & \cellcolor{gray!6}{1002.226} & \cellcolor{gray!6}{546.468}\\
State Population & 816 & 4646787.342 & 5010349.873\\
\cellcolor{gray!6}{Density} & \cellcolor{gray!6}{816} & \cellcolor{gray!6}{355.973} & \cellcolor{gray!6}{1408.250}\\
\bottomrule
\end{tabular}
\end{table}


As shown in the first panel, we have the arrest rates and crime rates for each of the nine categories analyzed by the original authors. We also include real per capita values for personal income, unemployment insurance, income maintenance, retirement payments, state population, and density.

\begin{table}[H]

\caption{\label{tab:tab:replicatetable2b}Demographic Variables Summary}
\centering
\begin{tabular}[t]{lrrr}
\toprule
Variable & N & Mean & Standard Deviation\\
\midrule
\cellcolor{gray!6}{White Male 1019} & \cellcolor{gray!6}{816} & \cellcolor{gray!6}{0.067} & \cellcolor{gray!6}{0.015}\\
Black Male 1019 & 816 & 0.010 & 0.011\\
\cellcolor{gray!6}{Other Male 1019} & \cellcolor{gray!6}{816} & \cellcolor{gray!6}{0.004} & \cellcolor{gray!6}{0.008}\\
White Female 1019 & 816 & 0.064 & 0.015\\
\cellcolor{gray!6}{Black Female 1019} & \cellcolor{gray!6}{816} & \cellcolor{gray!6}{0.010} & \cellcolor{gray!6}{0.011}\\
\addlinespace
Other Female 1019 & 816 & 0.004 & 0.007\\
\cellcolor{gray!6}{White Male 2029} & \cellcolor{gray!6}{816} & \cellcolor{gray!6}{0.074} & \cellcolor{gray!6}{0.012}\\
Black Male 2029 & 816 & 0.010 & 0.010\\
\cellcolor{gray!6}{Other Male 2029} & \cellcolor{gray!6}{816} & \cellcolor{gray!6}{0.004} & \cellcolor{gray!6}{0.007}\\
White Female 2029 & 816 & 0.073 & 0.012\\
\addlinespace
\cellcolor{gray!6}{Black Female 2029} & \cellcolor{gray!6}{816} & \cellcolor{gray!6}{0.010} & \cellcolor{gray!6}{0.012}\\
Other Female 2029 & 816 & 0.004 & 0.007\\
\cellcolor{gray!6}{White Male 3039} & \cellcolor{gray!6}{816} & \cellcolor{gray!6}{0.066} & \cellcolor{gray!6}{0.012}\\
Black Male 3039 & 816 & 0.007 & 0.008\\
\cellcolor{gray!6}{Other Male 3039} & \cellcolor{gray!6}{816} & \cellcolor{gray!6}{0.003} & \cellcolor{gray!6}{0.007}\\
\addlinespace
White Female 3039 & 816 & 0.066 & 0.012\\
\cellcolor{gray!6}{Black Female 3039} & \cellcolor{gray!6}{816} & \cellcolor{gray!6}{0.008} & \cellcolor{gray!6}{0.010}\\
Other Female 3039 & 816 & 0.003 & 0.007\\
\cellcolor{gray!6}{White Male 4049} & \cellcolor{gray!6}{816} & \cellcolor{gray!6}{0.048} & \cellcolor{gray!6}{0.009}\\
Black Male 4049 & 816 & 0.005 & 0.006\\
\addlinespace
\cellcolor{gray!6}{Other Male 4049} & \cellcolor{gray!6}{816} & \cellcolor{gray!6}{0.002} & \cellcolor{gray!6}{0.005}\\
White Female 4049 & 816 & 0.048 & 0.009\\
\cellcolor{gray!6}{Black Female 4049} & \cellcolor{gray!6}{816} & \cellcolor{gray!6}{0.005} & \cellcolor{gray!6}{0.007}\\
Other Female 4049 & 816 & 0.002 & 0.005\\
\cellcolor{gray!6}{White Male 5064} & \cellcolor{gray!6}{816} & \cellcolor{gray!6}{0.058} & \cellcolor{gray!6}{0.010}\\
\addlinespace
Black Male 5064 & 816 & 0.005 & 0.007\\
\cellcolor{gray!6}{Other Male 5064} & \cellcolor{gray!6}{816} & \cellcolor{gray!6}{0.002} & \cellcolor{gray!6}{0.006}\\
White Female 5064 & 816 & 0.062 & 0.012\\
\cellcolor{gray!6}{Black Female 5064} & \cellcolor{gray!6}{816} & \cellcolor{gray!6}{0.006} & \cellcolor{gray!6}{0.009}\\
Other Female 5064 & 816 & 0.002 & 0.007\\
\addlinespace
\cellcolor{gray!6}{White Male 65o} & \cellcolor{gray!6}{816} & \cellcolor{gray!6}{0.043} & \cellcolor{gray!6}{0.011}\\
Black Male 65o & 816 & 0.004 & 0.005\\
\cellcolor{gray!6}{Other Male 65o} & \cellcolor{gray!6}{816} & \cellcolor{gray!6}{0.001} & \cellcolor{gray!6}{0.005}\\
White Female 65o & 816 & 0.062 & 0.017\\
\cellcolor{gray!6}{Black Female 65o} & \cellcolor{gray!6}{816} & \cellcolor{gray!6}{0.005} & \cellcolor{gray!6}{0.008}\\
\addlinespace
Other Female 65o & 816 & 0.001 & 0.005\\
\bottomrule
\end{tabular}
\end{table}


In the second panel we present summary statistics for the demographic variables in the data. These features are represented as proportions of the whole, and are broken down by gender, white or black or other, and age group. 

\section{Empirical Model and Estimation}

This section presents each of the four methods applied to the data. First, we analyze using two-way fixed effects consistent with the authors original approach. Next, we implement the Bacon Decomposition. Finally, we implement the Callaway and Sant'anna estimator  and the Sun and Abraham event study estimator.

\subsection{Two way Fixed Effects}

For the two way fixed effects model, we use a similar specification to the authors. For each category of crime, we run an individual two way fixed effects regression where the natural log of crime rate is the outcome, and the covariates include the shall issue dummy variable, the arrest rate associated with that crime, and the various control variables related to economic and demographic conditions. To account for unobservable differences between the states and years in the data, we use allow for fixed effects on these two variables. The key difference between our implementation and the author's original is that we use state level data as opposed to county level data. We present the key results of this analysis in the tables below.


\begin{table}[htbp]
   \caption{\label{tab:replicatetable3a} Replication of Table 3 Panel A : Fixed Effects Regressions}
   \centering
   \small
   \begin{tabular}{lccc}
      \tabularnewline \midrule \midrule
      Dependent Variables:     & violent\_crime\_rate\_log    & property\_crime\_rate\_log    & murder\_crime\_rate\_log\\     
      Model:                   & (1)                          & (2)                           & (3)\\  
      \midrule
      \emph{Variables}\\
      shalll                   & -0.0978$^{***}$              & -0.0072                       & -0.0507\\   
                               & (0.0324)                     & (0.0194)                      & (0.0394)\\   
      Relevant\_Arrest\_Rate   & -0.0003                      & -0.0020$^{*}$                 & -0.0004\\   
                               & (0.0004)                     & (0.0011)                      & (0.0002)\\   
      \midrule
      \emph{Fixed-effects}\\
      state                    & Yes                          & Yes                           & Yes\\  
      year                     & Yes                          & Yes                           & Yes\\  
      \midrule
      \emph{Fit statistics}\\
      Observations             & 802                          & 809                           & 806\\  
      R$^2$                    & 0.98146                      & 0.96445                       & 0.94792\\  
      Within R$^2$             & 0.47334                      & 0.55426                       & 0.31137\\  
      \midrule \midrule
      \multicolumn{4}{l}{\emph{Clustered (state) standard-errors in parentheses}}\\
      \multicolumn{4}{l}{\emph{Signif. Codes: ***: 0.01, **: 0.05, *: 0.1}}\\
   \end{tabular}
   
   \par \raggedright 
   Control variables ommited from table, 
                           though they were included in the analysis. 
                           Consistent with the original paper, 
                           control variables include: density, personal\_income\_rpc, unemployment\_insurance\_rpc, income\_maintenance\_rpc, retirement\_payments\_rpc, state\_population, ppwm1019, ppbm1019, ppnm1019, ppwf1019, ppbf1019, ppnf1019, ppwm2029, ppbm2029, ppnm2029, ppwf2029, ppbf2029, ppnf2029, ppwm3039, ppbm3039, ppnm3039, ppwf3039, ppbf3039, ppnf3039, ppwm4049, ppbm4049, ppnm4049, ppwf4049, ppbf4049, ppnf4049, ppwm5064, ppbm5064, ppnm5064, ppwf5064, ppbf5064, ppnf5064, ppwm65o, ppbm65o, ppnm65o, ppwf65o, ppbf65o, ppnf65o.
\end{table}





\begin{table}[htbp]
   \caption{\label{tab:replicatetable3b} Replication of Table 3 Panel B : Fixed Effects Regressions}
   \centering
   \small
   \begin{tabular}{lccc}
      \tabularnewline \midrule \midrule
      Dependent Variables:     & rape\_crime\_rate\_log    & assault\_crime\_rate\_log    & robbery\_crime\_rate\_log\\     
      Model:                   & (1)                       & (2)                          & (3)\\  
      \midrule
      \emph{Variables}\\
      shalll                   & -0.0340                   & -0.1004$^{**}$               & -0.0532\\   
                               & (0.0395)                  & (0.0424)                     & (0.0439)\\   
      Relevant\_Arrest\_Rate   & -0.0006                   & -0.0028$^{***}$              & -0.0014\\   
                               & (0.0005)                  & (0.0009)                     & (0.0009)\\   
      \midrule
      \emph{Fixed-effects}\\
      state                    & Yes                       & Yes                          & Yes\\  
      year                     & Yes                       & Yes                          & Yes\\  
      \midrule
      \emph{Fit statistics}\\
      Observations             & 799                       & 809                          & 808\\  
      R$^2$                    & 0.94240                   & 0.96632                      & 0.98389\\  
      Within R$^2$             & 0.52351                   & 0.51645                      & 0.50471\\  
      \midrule \midrule
      \multicolumn{4}{l}{\emph{Clustered (state) standard-errors in parentheses}}\\
      \multicolumn{4}{l}{\emph{Signif. Codes: ***: 0.01, **: 0.05, *: 0.1}}\\
   \end{tabular}
   
   \par \raggedright 
   Control variables ommited from table, 
                           though they were included in the analysis. 
                           Consistent with the original paper, 
                           control variables include: density, personal\_income\_rpc, unemployment\_insurance\_rpc, income\_maintenance\_rpc, retirement\_payments\_rpc, state\_population, ppwm1019, ppbm1019, ppnm1019, ppwf1019, ppbf1019, ppnf1019, ppwm2029, ppbm2029, ppnm2029, ppwf2029, ppbf2029, ppnf2029, ppwm3039, ppbm3039, ppnm3039, ppwf3039, ppbf3039, ppnf3039, ppwm4049, ppbm4049, ppnm4049, ppwf4049, ppbf4049, ppnf4049, ppwm5064, ppbm5064, ppnm5064, ppwf5064, ppbf5064, ppnf5064, ppwm65o, ppbm65o, ppnm65o, ppwf65o, ppbf65o, ppnf65o.
\end{table}





\begin{table}[htbp]
   \caption{\label{tab:replicatetable3c} Replication of Table 3 Panel C : Fixed Effects Regressions}
   \centering
   \small
   \begin{tabular}{lccc}
      \tabularnewline \midrule \midrule
      Dependent Variables:     & burglary\_crime\_rate\_log    & larceny\_crime\_rate\_log    & autotheft\_crime\_rate\_log\\     
      Model:                   & (1)                           & (2)                          & (3)\\  
      \midrule
      \emph{Variables}\\
      shalll                   & -0.0461$^{**}$                & 0.0033                       & -0.0090\\   
                               & (0.0190)                      & (0.0138)                     & (0.0283)\\   
      Relevant\_Arrest\_Rate   & -0.0052$^{***}$               & -0.0011                      & -0.0003$^{*}$\\   
                               & (0.0015)                      & (0.0010)                     & (0.0002)\\   
      \midrule
      \emph{Fixed-effects}\\
      state                    & Yes                           & Yes                          & Yes\\  
      year                     & Yes                           & Yes                          & Yes\\  
      \midrule
      \emph{Fit statistics}\\
      Observations             & 809                           & 809                          & 808\\  
      R$^2$                    & 0.95597                       & 0.96600                      & 0.96116\\  
      Within R$^2$             & 0.50429                       & 0.54529                      & 0.60312\\  
      \midrule \midrule
      \multicolumn{4}{l}{\emph{Heteroskedasticity-robust standard-errors in parentheses}}\\
      \multicolumn{4}{l}{\emph{Signif. Codes: ***: 0.01, **: 0.05, *: 0.1}}\\
   \end{tabular}
   
   \par \raggedright 
   Control variables ommited from table, 
                           though they were included in the analysis. 
                           Consistent with the original paper, 
                           control variables include: density, personal\_income\_rpc, unemployment\_insurance\_rpc, income\_maintenance\_rpc, retirement\_payments\_rpc, state\_population, ppwm1019, ppbm1019, ppnm1019, ppwf1019, ppbf1019, ppnf1019, ppwm2029, ppbm2029, ppnm2029, ppwf2029, ppbf2029, ppnf2029, ppwm3039, ppbm3039, ppnm3039, ppwf3039, ppbf3039, ppnf3039, ppwm4049, ppbm4049, ppnm4049, ppwf4049, ppbf4049, ppnf4049, ppwm5064, ppbm5064, ppnm5064, ppwf5064, ppbf5064, ppnf5064, ppwm65o, ppbm65o, ppnm65o, ppwf65o, ppbf65o, ppnf65o.
\end{table}




As shown in the tables, we find a negative relationship between the shall issue dummy variable and the log of crime rate for all crime categories besides larceny. This result would indicate that implementation of a shall issue law should decrease crime rates for all categories besides larceny. The two largest coefficients (in terms of order of magnitude) appear in the regression for violent crime (column 1 of Table 4) and aggravated assault (column 2 of Table 5). Both of these coefficients are statistically significant at a 1 percent level, and suggest the largest change in crime rate for a change to a shall issue law setting.

\newpage

\subsection{Bacon Decomposition}

\begin{itemize}
\item
Explain Analysis: Bacon Decomposition without controls
\item
Explain weights and DiD estimates
\item
Explain early to late 2x2s and late to early 2x2s. What is the problem with late to early?
\end{itemize}

\begin{table}[H]

\caption{\label{tab:tab:bacondecompositionAssault}Bacon Decomposition - Assault Crime Rate Log}
\centering
\begin{tabular}[t]{lrrr}
\toprule
Type & Average Estimate & Group Weight & Weighted Estimate\\
\midrule
\cellcolor{gray!6}{Earlier vs Later Treated} & \cellcolor{gray!6}{0.149} & \cellcolor{gray!6}{0.068} & \cellcolor{gray!6}{0.008}\\
Later vs Always Treated & -0.031 & 0.159 & 0.001\\
\cellcolor{gray!6}{Later vs Earlier Treated} & \cellcolor{gray!6}{-0.017} & \cellcolor{gray!6}{0.023} & \cellcolor{gray!6}{-0.003}\\
Treated vs Untreated & -0.221 & 0.749 & -0.138\\
\cellcolor{gray!6}{Total TWFE} & \cellcolor{gray!6}{NaN} & \cellcolor{gray!6}{NaN} & \cellcolor{gray!6}{-0.132}\\
\bottomrule
\end{tabular}
\end{table}


\begin{table}[H]

\caption{\label{tab:tab:bacondecompositionAutotheft}Bacon Decomposition - Autotheft Crime Rate Log}
\centering
\begin{tabular}[t]{lrrr}
\toprule
Type & Average Estimate & Group Weight & Weighted Estimate\\
\midrule
\cellcolor{gray!6}{Earlier vs Later Treated} & \cellcolor{gray!6}{0.089} & \cellcolor{gray!6}{0.068} & \cellcolor{gray!6}{0.006}\\
Later vs Always Treated & 0.189 & 0.159 & 0.034\\
\cellcolor{gray!6}{Later vs Earlier Treated} & \cellcolor{gray!6}{0.099} & \cellcolor{gray!6}{0.023} & \cellcolor{gray!6}{0.002}\\
Treated vs Untreated & 0.009 & 0.749 & 0.026\\
\cellcolor{gray!6}{Total TWFE} & \cellcolor{gray!6}{NaN} & \cellcolor{gray!6}{NaN} & \cellcolor{gray!6}{0.068}\\
\bottomrule
\end{tabular}
\end{table}


\begin{table}[H]

\caption{\label{tab:tab:bacondecompositionBurglary}Bacon Decomposition - Burglary Crime Rate Log}
\centering
\begin{tabular}[t]{lrrr}
\toprule
Type & Average Estimate & Group Weight & Weighted Estimate\\
\midrule
\cellcolor{gray!6}{Earlier vs Later Treated} & \cellcolor{gray!6}{-0.015} & \cellcolor{gray!6}{0.068} & \cellcolor{gray!6}{-0.002}\\
Later vs Always Treated & 0.017 & 0.159 & 0.005\\
\cellcolor{gray!6}{Later vs Earlier Treated} & \cellcolor{gray!6}{-0.016} & \cellcolor{gray!6}{0.023} & \cellcolor{gray!6}{-0.001}\\
Treated vs Untreated & -0.007 & 0.749 & 0.006\\
\cellcolor{gray!6}{Total TWFE} & \cellcolor{gray!6}{NaN} & \cellcolor{gray!6}{NaN} & \cellcolor{gray!6}{0.008}\\
\bottomrule
\end{tabular}
\end{table}


\begin{table}[H]

\caption{\label{tab:tab:bacondecompositionLarceny}Bacon Decomposition - Larceny Crime Rate Log}
\centering
\begin{tabular}[t]{lrrr}
\toprule
Type & Average Estimate & Group Weight & Weighted Estimate\\
\midrule
\cellcolor{gray!6}{Earlier vs Later Treated} & \cellcolor{gray!6}{0.0088734} & \cellcolor{gray!6}{0.0683810} & \cellcolor{gray!6}{-0.0004160}\\
Later vs Always Treated & 0.0477985 & 0.1589397 & 0.0080003\\
\cellcolor{gray!6}{Later vs Earlier Treated} & \cellcolor{gray!6}{0.0616798} & \cellcolor{gray!6}{0.0233921} & \cellcolor{gray!6}{0.0004859}\\
Treated vs Untreated & 0.0346292 & 0.7492871 & 0.0280699\\
\cellcolor{gray!6}{Total TWFE} & \cellcolor{gray!6}{NaN} & \cellcolor{gray!6}{NaN} & \cellcolor{gray!6}{0.0361400}\\
\bottomrule
\end{tabular}
\end{table}


\begin{table}[H]

\caption{\label{tab:tab:bacondecompositionRape}Bacon Decomposition: Rape Crime Rate Log}
\centering
\begin{tabular}[t]{lrrr}
\toprule
Type & Average Estimate & Group Weight & Weighted Estimate\\
\midrule
\cellcolor{gray!6}{Earlier vs Later Treated} & \cellcolor{gray!6}{-0.0264193} & \cellcolor{gray!6}{0.0683810} & \cellcolor{gray!6}{-0.0026425}\\
Later vs Always Treated & -0.2001560 & 0.1589397 & -0.0305273\\
\cellcolor{gray!6}{Later vs Earlier Treated} & \cellcolor{gray!6}{-0.0106483} & \cellcolor{gray!6}{0.0233921} & \cellcolor{gray!6}{-0.0019283}\\
Treated vs Untreated & -0.0030557 & 0.7492871 & 0.0030423\\
\cellcolor{gray!6}{Total TWFE} & \cellcolor{gray!6}{NaN} & \cellcolor{gray!6}{NaN} & \cellcolor{gray!6}{-0.0320558}\\
\bottomrule
\end{tabular}
\end{table}


\begin{table}[H]

\caption{\label{tab:tab:bacondecompositionRobbery}Bacon Decomposition - Robbery Crime Rate Log}
\centering
\begin{tabular}[t]{lrrr}
\toprule
Type & Average Estimate & Group Weight & Weighted Estimate\\
\midrule
\cellcolor{gray!6}{Earlier vs Later Treated} & \cellcolor{gray!6}{0.0961047} & \cellcolor{gray!6}{0.0683810} & \cellcolor{gray!6}{0.0073682}\\
Later vs Always Treated & 0.0940074 & 0.1589397 & 0.0174822\\
\cellcolor{gray!6}{Later vs Earlier Treated} & \cellcolor{gray!6}{0.1505417} & \cellcolor{gray!6}{0.0233921} & \cellcolor{gray!6}{0.0020947}\\
Treated vs Untreated & -0.0263060 & 0.7492871 & -0.0100514\\
\cellcolor{gray!6}{Total TWFE} & \cellcolor{gray!6}{NaN} & \cellcolor{gray!6}{NaN} & \cellcolor{gray!6}{0.0168936}\\
\bottomrule
\end{tabular}
\end{table}


\begin{table}[H]

\caption{\label{tab:tab:bacondecompositionMurder}Bacon Decomposition - Murder Crime Rate Log}
\centering
\begin{tabular}[t]{lrrr}
\toprule
Type & Average Estimate & Group Weight & Weighted Estimate\\
\midrule
\cellcolor{gray!6}{Earlier vs Later Treated} & \cellcolor{gray!6}{0.0545024} & \cellcolor{gray!6}{0.0683810} & \cellcolor{gray!6}{0.0054525}\\
Later vs Always Treated & -0.0384771 & 0.1589397 & -0.0012542\\
\cellcolor{gray!6}{Later vs Earlier Treated} & \cellcolor{gray!6}{0.0188246} & \cellcolor{gray!6}{0.0233921} & \cellcolor{gray!6}{0.0000418}\\
Treated vs Untreated & -0.0848318 & 0.7492871 & -0.0415959\\
\cellcolor{gray!6}{Total TWFE} & \cellcolor{gray!6}{NaN} & \cellcolor{gray!6}{NaN} & \cellcolor{gray!6}{-0.0373558}\\
\bottomrule
\end{tabular}
\end{table}


\begin{table}[H]

\caption{\label{tab:tab:bacondecompositionProperty}Bacon Decomposition: Property Crime Rate Log}
\centering
\begin{tabular}[t]{lrrr}
\toprule
Type & Average Estimate & Group Weight & Weighted Estimate\\
\midrule
\cellcolor{gray!6}{Earlier vs Later Treated} & \cellcolor{gray!6}{0.0036098} & \cellcolor{gray!6}{0.0683810} & \cellcolor{gray!6}{-0.0007191}\\
Later vs Always Treated & 0.0504242 & 0.1589397 & 0.0085564\\
\cellcolor{gray!6}{Later vs Earlier Treated} & \cellcolor{gray!6}{0.0490119} & \cellcolor{gray!6}{0.0233921} & \cellcolor{gray!6}{0.0001507}\\
Treated vs Untreated & 0.0238073 & 0.7492871 & 0.0210569\\
\cellcolor{gray!6}{Total TWFE} & \cellcolor{gray!6}{NaN} & \cellcolor{gray!6}{NaN} & \cellcolor{gray!6}{0.0290449}\\
\bottomrule
\end{tabular}
\end{table}


\begin{table}[H]

\caption{\label{tab:tab:bacondecompositionViolent}Bacon Decomposition: Violent Crime Rate Log}
\centering
\begin{tabular}[t]{lrrr}
\toprule
Type & Average Estimate & Group Weight & Weighted Estimate\\
\midrule
\cellcolor{gray!6}{Earlier vs Later Treated} & \cellcolor{gray!6}{0.1000224} & \cellcolor{gray!6}{0.0683810} & \cellcolor{gray!6}{0.0051705}\\
Later vs Always Treated & -0.0596393 & 0.1589397 & -0.0044690\\
\cellcolor{gray!6}{Later vs Earlier Treated} & \cellcolor{gray!6}{0.0208665} & \cellcolor{gray!6}{0.0233921} & \cellcolor{gray!6}{-0.0017883}\\
Treated vs Untreated & -0.1422163 & 0.7492871 & -0.0839315\\
\cellcolor{gray!6}{Total TWFE} & \cellcolor{gray!6}{NaN} & \cellcolor{gray!6}{NaN} & \cellcolor{gray!6}{-0.0850183}\\
\bottomrule
\end{tabular}
\end{table}


\subsection{Callaway and Sant'anna}

\textbf{Prompt: } Present a subsection in which you implement the Callaway and Sant’anna estimator.  Describe the model with an equation and a description (short). Use the double robust specification. You will be analyzing each outcome and reporting the overall ATT.  Do not report the group-level ATTs because many states simply do not have enough states per treatment date for the bootstrapping to provide accurate 95 percent confidence intervals.  Use no more than 2 covariates – use your own judgment in selectin them. Report this in Table 4 and in your discussion compare what you found with the original findings.  Are they similar?  If not how do they differ?

\subsection{Event Study (Sun and Abraham)}

\textbf{Prompt: } Finally, implement the Sun and Abraham event study.  While you can estimate Callaway and Sant’anna event studies, I would like to use Sun and Abraham.  Explain the interaction weighted estimator and show a figure of each crime.  Do pretrends appear to hold?  How confident do you feel then that parallel trends holds for each outcome.  This should only be presented as a Figure, not a table.  

\section{Conclusion}

\textbf{Prompt: }What do you think you learned from this exercise?  Feel free to discuss as little or as much as you want.  I am just interested in your opinions.  The purpose of this is merely to give you a nudge in considering how to interpret results and offer some commentary. 

\end{document}

